\documentclass{adsrprt}

% change this: (do NOT add your name, only your s-number!)
\newcommand\snumA{s1234567}        % student number 1
\newcommand\snumB{s7654321}        % student number 2
\newcommand\assgnnum{2}            % assignment number
\newcommand\assgntitle{Equations}  % assignment title

% do not change this:
\title{Assignment \assgnnum: ``\assgntitle''\\Programming report}
\subtitle{\snumA\ and \snumB\\Algorithms and Data Structures in C (2024-2025)}
\headertitle{Assignment \assgnnum: ``\assgntitle''}
\headersubtitle{Report by \snumA\ and \snumB}


\begin{document}
\maketitle

%%%%%%%%%%%%%%%%%%%%%%%%%%%%%%%%%%%%%%%%%%%%%%%%%%%%%%%%%%%%%%%%%%%
% NOTE: You MUST read and follow Appendix E of the lecture notes! %
%%%%%%%%%%%%%%%%%%%%%%%%%%%%%%%%%%%%%%%%%%%%%%%%%%%%%%%%%%%%%%%%%%%

\section{Problem description}

...

\section{Problem analysis}

% The following example illustrates how to typeset pseudocode:

\begin{tabbing}
xxx \= xxx \= xxx \= xxx \= \kill
\Algorithm~Downheap(v) \\
\> \Input: node v in a heap, with possibly a conflict \\
\> \> with the heap order between v and its children \\
\> \Result: heap order is restored \\
\> \If~v has at least one child \Then \\
\> \> lc \becomes~the left child of v \\
\> \> rc \becomes~the right child of v (or lc, when v has no right child) \\
\> \> \If~(value of lc) \(>\) (value of v) and (value of lc) $>$ (value of rc) \Then \\
\> \> \> swap the values of lc and v \\
\> \> \> Downheap(lc) \\
\> \> \Else~\If~(value of rc) \(>\) (value of v) \Then \\
\> \> \> \Comment{ now also (value of rc) $\geq$ (value of lc) } \\
\> \> \> swap the values of rc and v \\
\> \> \> Downheap(rc)
\end{tabbing}

\section{Program design}

...

\section{Evaluation of the program}

...

% \section{Extension of the program} % Optional
%
% ...

\section{Process description}

...

\section{Conclusions}

...

\section{Appendix: program text}

% Here you should include the program text.
% Do NOT use screenshots or similar methods.
% Below you can see how to use \lstinputlisting{}.

\texttt{\bfseries graph.h}
\lstinputlisting{graph.h}

\texttt{\bfseries graph.c}
\lstinputlisting{graph.c}

\texttt{\bfseries solver.h}
\lstinputlisting{solver.h}

\texttt{\bfseries solver.c}
\lstinputlisting{solver.c}

\texttt{\bfseries main.c}
\lstinputlisting{main.c}

% \section{Appendix: test cases} % Optional
%
% ...

% \section{Appendix: Extensions} % Optional
%
% ...

\end{document}
